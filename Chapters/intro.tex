\documentclass[../main.tex]{subfiles}
\begin{document}

Over the past decade exoplanet observation programs and population surveys have discovered that exoplanets are ubiquitous and their masses, sizes, and system architectures are diverse. The exoplanet community seeks to understand which processes and environments inform this diversity of planet/system types and how this relates to their atmospheric structure and composition. Ultimately, we are searching for the signatures of life and the prevalence of conditions conducive to life.

\section{Direct Imaging}

Direct imaging is a challenging exoplanet discovery and characterization technique due to the extreme contrast in intensity between the stellar host and planetary companion and the need to spatially resolve them. Sophisticated ground-based high contrast instruments have been or are being built including: the Gemini Planet Imager \parencite[GPI,][]{Macintosh2008, Macintosh2014}, SPHERE at VLT \parencite{Carbillet2011}, SCExAO at Subaru \parencite{Jovanovic2015}, P1640 \parencite{Crepp2011} and the Stellar Double Coronagraph \parencite[SDC,][]{Mawet2014} at Palomar, the Keck Planet Imager and Characterizer (KPIC) at Keck \parencite{Mawet2016}, and MagAO-X at the Magellan Clay Telescope \parencite{Males2018}. Around the nearest stars we are able to survey formation conditions in primordial disks and the temperature of a handful of giant young exoplanets in the stellar neighborhood. However, the advent of high-speed low-noise cameras in the near-IR and post-coronagraphic fiber couplers for spectroscopy will significantly increase the potential of direct imaging. 



Direct imaging is a challenging exoplanet discovery and characterization technique due to the extreme contrast ($<$10$^{-4}$ for ground based targets) and small angular separations ($\lesssim$1$^{\prime\prime}$) between the planetary companion and its stellar host. Despite this, adaptive optics (AO) and coronagraphy have enabled the discovery of planets up to $\sim$10$^6$ times fainter than their host stars \parencite{Marois+Macintosh+Barman+etal_2008, Lagrange+Bonnefoy+Chauvin+etal_2010, Kuzuhara+Tamura+Kudo+etal_2013, Macintosh+Graham+Barman+etal_2015, Keppler+Benisty+Muller+etal_2018}. Imaging an exoplanet requires subtracting the light of its host star in the form of the point-spread function (PSF). If this background were static and could be subtracted perfectly, exoplanet imaging would be limited only by the photon shot noise of the bright host star. Instead, high-contrast imaging is limited by uncontrolled scattered and diffracted light, which produces a coherent speckle halo in the image plane \parencite{Guyon_2005}.   

Fast atmospheric speckles average down over an observation, while slower, quasistatic speckles must be removed using post-processing techniques. Angular differential imaging \parencite[ADI,][]{Marois+Lafreniere+Doyon+etal_2006} exploits the rotation of the Earth, and hence the field-of-view of an altitude-azimuth telescope, to distinguish diffraction speckles from astrophysical sources. Spectral differential imaging \parencite[SDI,][]{Racine+Walker+Nadeau+etal_1999,Marois+Doyon+Racine+etal_2000,Sparks+Ford_2002} uses the scaling of diffraction speckles with wavelength. Since the initial development of ADI and SDI, a variety of post-processing algorithms have refined their approaches to dig deeper into the stellar PSF \parencite[e.g.][]{Lafreniere+Marois+Doyon+etal_2007,Soummer+Pueyo+Larkin_2012,Marois+Correia+Galicher+etal_2014}.

The time variability and chromaticity of quasi-static speckles limit the performance of ADI and SDI \parencite{Gerard+Marois+Currie+etal_2019}. Both techniques also suffer at small separations where exoplanets are more likely to hide. The speckle spectral dispersion used by SDI is proportional to the separation: close to the star, it becomes smaller than the planet's PSF. For ADI, the arclength traced by the companion's sky rotation is proportional to the separation. Furthermore, the precision of the background estimate for PSF subtraction is limited by low counting statistics at small separations \parencite{Mawet_2014}. Even without these issues, the variability induced by speckle fluctuations can dominate the photon noise and be well above the shot noise expected from the total number of photons. 

\end{document}